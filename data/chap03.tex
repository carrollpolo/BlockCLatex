% !TeX root = ../thuthesis-example.tex

\chapter{区块链的技术构成}

\section{分布式账本技术}

定义:分布式账本(Distributed ledger)是一种在网络成员之间共享、复制和同步的数据库。
用处:分布式账本记录网络参与者之间的交易,比如资产或数据的交换。这种共享账本消除了调解不同账本的时间和开支 
实质:分布式账本,从实质上说就是一个可以在多个站点、不同地理位置或者多个机构组成的网络里进行分享的资产数据库。在一个网络里的参与者可以获得一个唯一、真实账本的副本。账本里的任何改动都会在所有的副本中被反映出来,反应时间会在几分钟甚至是几秒内。在这个账本里存储的资产可以是金融、法律定义上的、实体的或是电子的资产。在这个账本里存储的资产的安全性和准确性是通过公私钥以及签名的使用去控制账本的访问权,从而实现密码学基础上的维护。根据网络中达成共识的规则,账本中的记录可以由一个、一些或者是所有参与者共同进行更新。


\section{共识机制剖析}

机制定义:共识机制是一整套由协议、激励和想法构成的体系[3]

机制好处:使得整个网络的节点能够就区块链状态达成一致。

剖析:共识机制是一个程序,通过这个程序,区块链网络的所有对等人就分布式账本的当前状态达成共同协议。通过这种方式,共识机制实现了区块链网络的可靠性,并在分布式计算环境中的未知对等体之间建立了信任。从本质上讲,共识机制确保每一个添加到区块链的新区块都是区块链中所有节点同意的唯一版本的真相。区块链共识协议包括一些具体的目标,如达成协议、协作、合作、每个节点的平等权利,以及每个节点在共识过程中的强制性参与。因此,共识机制的目的是找到一个共同的协议,对整个网络来说是一个胜利。

\section{加密算法原理}

加密算法类型大体可以分为三类:对称加密、非对称加密、单向加密。
对称加密:加密和解密都用相同的密钥[4]
非对称加密:采用公钥和私钥两种不同的密码来进行加解密。公钥和私钥是成对存在,公钥是从私钥中提取产生公开给所有人的,如果使用公钥对数据进行加密,那么只有对应的私钥(不能公开)才能解密,反之亦然。N 个用户通信,需要2N个密钥。[4]
单向加密:发送者将明文通过单向加密算法加密生成定长的密文串,然后传递给接收方,接收方将用于比对验证的明文使用相同的单向加密算法进行加密,得出加密后的密文串。[4]

