\chapter{区块链的应用实践}

\section{金融领域的应用}

金融交易: 区块链技术与金融市场应用有非常高的契合度. 区块链可以在去中心化系统中自发地产生信用, 能够建立无中心机构信用背书的金融市场, 从而在很大程度上实现了 “金融脱媒”, 这对第三方支付、资金托管等存在中介机构的商业模式来说是颠覆性的变革; 在互联网金融领域, 区块链特别适合或者已经应用于股权众筹、P2P 网络借贷和互联网保险等商业模式; 

证券和银行业务也是区块链的重要应用领域, 传统证券交易需要经过中央结算机构、银行、证券公司和交易所等中心机构的多重协调, 而利用区块链自动化智能合约和可编程的特点, 能够极大地降低成本和提高效率, 避免繁琐的中心化清算交割过程, 实现方便快捷的金融产品交易; 同时, 区块链和比特币的即时到帐的特点可使得银行实现比 SWIFT 代码体系更为快捷、经济和安全的跨境转账; 这也是目前 R3CEV 和纳斯达克等各大银行、证券商和金融机构相继投入区块链技术研发的重要原因。


\section{选举投票的应用}

选举投票: 投票是区块链技术在政治事务中的代表性应用. 基于区块链的分布式共识验证、不可篡改等特点, 可以低成本高效地实现政治选举、企业股东投票等应用; 同时, 区块链也支持用户个体对特定议题的投票. 例如, 通过记录用户对特定事件是否发生的投票, 可以将区块链应用于博彩和预测市场等场景[11]; 通过记录用户对特定产品的投票评分与建议, 可以实现大规模用户众包设计产品的 “社会制造” 模式等。


\section{数据鉴证的应用}

数据鉴证: 区块链数据带有时间戳、由共识节点共同验证和记录、不可篡改和伪造,应用于各类数据公证和审计场景. 例如, 区块链可以永久地安全存储由政府机构核发的各类许可证、登记、证明、认证和记录等, 并可在任意时间方便地证明某项数据的存在性和一定程度上的真实性. 包括德勤在内的多家审计公司已经部署区块链技术来帮助其审计师实现低成本和高效地实时审计; Factom公司则基于区块链设计了一套准确的、可核查的和不可更改的审计公证流程与方法[12].

