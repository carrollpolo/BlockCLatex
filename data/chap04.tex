% !TeX root = ../thuthesis-example.tex

\chapter{区块链的常见算法}


\section{哈希算法及其应用}

定义:也称为散列算法,是一种从任意长度的数据中创建小的数字“指纹”的方法。

特点:无论数据的长度如何,哈希算法的输出长度都是固定的,并且哈希算法是单向的,这意味着从哈希值不能反向推导出原始数据。

应用:

文件校验:通过比较文件的哈希值,可以验证文件在传输过程中是否被篡改。

数字签名:哈希值可以用于验证文档的完整性,确保文档未被篡改。

密码存储:出于安全考虑,系统通常不会存储用户的明文密码,而是存储密码的哈希值。在用户登录时,系统会将用户输入的密码进行哈希处理,然后与存储的哈希值进行比较。


\section{数字签名算法介绍}

概述:数字签名是一种用于验证数字文档完整性、真实性和身份的加密技术。它基于公钥密码学原理,结合哈希函数的使用,提供了一种确保数字信息安全的方法。

工作原理:数字签名的工作过程涉及两个关键的密钥:私钥和公钥。私钥是由文档的签名者所拥有,并且必须保密。公钥则是公开的,任何人都可以访问。签名过程包括以下步骤:

1.	签名者首先使用哈希函数对文档进行摘要计算,生成一个固定长度的哈希值。

2.	然后,签名者使用自己的私钥对这个哈希值进行加密,生成数字签名。

3.	数字签名随文档一起发送给接收者。

4.	接收者使用签名者的公钥对数字签名进行解密,得到文档的哈希值。

5.	接收者再次使用相同的哈希函数对文档进行摘要计算,得到一个新的哈希值。

6.	如果两个哈希值相等,则证明文档未被篡改,签名者的身份得到验证。


\section{共识算法的比较与选择[5]}

\subsection{工作量证明}

工作量证明(PoW,Proof of Work)

优点:自 2009 年以来得到了广泛测试,目前依然得到广泛的使用。

缺点:速度慢;耗能巨大,对环境不好;易受“规模经济”(economies of scale)的影响。

\subsection{权益证明}

权益证明(PoS,Proof of Stake)

优点:节能;攻击者代价更大;不易受“规模经济”的影响。

缺点:「无利害关系」(Nothing at stake)攻击问题。

\subsection{延迟工作量证明}

延迟工作量证明(dPoW,Delayed Proof-of-Work)

优点:节能;安全性增加;可以通过非直接提供 Bitcoin(或是其它任何安全链),添加价值到其它区块链,无需付出 Bitcoin(或是其它任何安全链)交易的代价。

缺点:只有使用PoW或PoS的区块链,才能采用这种共识算法;在“公证员激活”(Notaries Active)模式下,必须校准不同节点(公证员或正常节点)的哈希率,否则哈希率间的差异会爆炸(下文将给出详细解释)。

\subsection{授权 PoS}

授权 PoS(DPoS,Delegated Proof-of-Stake)

优点:节能;快速;高流量博客网站 Steemit 就使用了它。EOS 的块时间是 0.5 秒。

缺点:略为中心化;拥有高权益的参与者可投票使自己成为一名验证者(这是近期已在 EOS 中出现的问题)。

\subsection{实用拜占庭容错算法}

实用拜占庭容错算法(PBFT:Practical Byzantine Fault Tolerance)

优点:高速、可扩展。

缺点:通常用于私有网络和许可网络。

\subsection{授权拜占庭容错算法}

授权拜占庭容错算法(dBFT,Delegated Byzantine Fault Tolerance)

优点:快速;可扩展。

缺点:每个人都争相成为根链。其中可能存在多个根链。

\subsection{权威证明}

权威证明(PoA,Proof-of-Authority)

优点:节能、快速。

缺点:略为中心化;虽然可用于公有区块链,但是通常用于私有区块链和许可区块链。

\subsection{所用时间证明}

所用时间证明(PoET,Proof of Elapsed Time)

优点:参与代价低;更多人可轻易加入,进而达到去中心化;对于所有参与者而言,更易于验证领导者是通过合法选举产生的;控制领导者选举过程的代价,是与从中获得的价值成正比的。

缺点:尽管 PoET 的代价低,但是必须要使用特定的硬件;不适用于公有区块链。

\subsection{权益流通证明}

权益流通证明(PoSV,Proof of Stake Velocity)

PoSV 是作为 PoW 和 PoS 的一种替代方法而提出的,其目的是提高 P2P 网络的安全性,进而用于确认 Reddcoin 交易。

\subsection{恒星共识}

优点:去中心化控制;低延迟;灵活的信任机制;渐进安全(Asymptotic security)。
