% !TeX root = ../thuthesis-example.tex

\chapter{区块链的意义}

\section{对数据安全与隐私保护的意义}

数据安全与隐私保护的现状堪忧:

1、(隐私数据方面)2024年上半年全网监测并分析验证有效的数据泄露事件16011起,较2023年下半年增长59.58%,共9539起[2];
2、(隐私数据方面)2024年上半年监测到3.4万个黑产团伙中,经分析验证涉及真实数据泄露事件的黑产团伙共计1973个,较2023年下半年增新增984个,增长近一倍[2];
3、(交易方面)从行业分布来看,2024年上半年数据泄露事件涉及85个行业,数据泄露事件数量Top5行业分别为银行、电商、消费金融、保险、快递[2]。;
4、(隐私数据方面)针对黑产数据交易市场研究发现,2024年上半年利用Facetime诈骗活动增多,2024年上半年泄露的数据中包含“IOS”字段的相关风险事件高达1237起,较2023年下半年增加8倍[2]。

隐私数据方面:区块链技术通过去中心化的特点增强了数据安全性,传统的数据存储通常集中在中心化的服务器上,这使得数据容易受到黑客攻击和数据篡改。而区块链技术将数据分布在网络的各个节点上,每一块数据都是通过加密算法链接在一起形成链式结构。这种去中心化的特点使得黑客很难入侵所有节点并篡改数据,因此增加了数据的安全性。

交易方面:区块链技术采用了强大的加密算法,保证了数据传输和存储的安全性每一笔交易都经过加密,只有特定的私钥持有者才能解密并访问数据。这使得数据在传输和存储中受到了强大的保护,即使数据被截获或泄露,黑客也无法获得有用的信息。同时,区块链上的数据只能被授权的用户修改和访问,其他人无法篡改或删除数据,确保了数据的完整性。

数据的匿名和隐私保护方面:传统的数据库存储个人信息或敏感数据时,通常需要用户提供真实姓名、身份证号码等个人身份信息。而区块链上的智能合约通过匿名特性,使得用户可以在不暴露真实身份的情况下进行交易和数据分享。智能合约使用密钥对验证身份,而无需暴露个人身份信息,确保了用户的隐私和数据的安全。

数据共享和溯源方面:传统的数据共享存在隐私泄露和数据篡改的问题,而区块链技术通过确保数据的真实性、不可篡改性和可追溯性,建立了可信任的数据共享机制。利用区块链技术,用户的个人数据可以得到充分保护的同时,还能够与其他用户安全地分享数据,促进信息的共享和交流。


%\begin{figure}
%  \centering
%  \includegraphics[width=0.5\linewidth]{example-image-a.pdf}
%  \caption*{国外的期刊习惯将图表的标题和说明文字写成一段,需要改写为标题只含图表的名称,其他说明文字以注释方式写在图表下方,或者写在正文中。}
%  \caption{示例图片标题}
%  \label{fig:example}
%\end{figure}

\section{对商业信任机制的影响}

在经济学的语境里,信任是一种行为策略,而信用可以被看做一个主观概率水平。只要潜在收益与守信行为的概率的乘积大于潜在损失与不守信的概率的乘积,信用就是占优势的行为策略。这是一个简单的信用表达式,但是可操作性很低。人并不是绝对理性的,而且不同的社会环境和制度会影响主体的选择。因此,难度最高的是构造出可以精确计算出行为的可信程度的系统。

但区块链的优越性正在于,通过精巧的设计,在这个系统内部,违约成本和预期收益可以被精确计算,因此理性个体将在权衡利弊后丧失欺诈的动力,从而一定程度优化商业信任机制。

\section{对社会和经济体系的潜在变革}

\subsection{对社会的潜在变革}

便捷司法存证,促进社会正义:

难点:在司法中,与传统司法证据相比,电子证据等的获取具有以下难点。
1.取证成本高。当前司法取证依赖于具有司法机制的存证机构,具有取证周期长、费用高等特点。同时人力投入大,操作成本较高。
2.取证难校验,公信力可能不足。由于电子证据本身易篡改、难溯源的特点,电子取证的权威性依赖于取证机构的资质与公信力,且取证后难以校验、追责。
2018年,我国公布了《最高人民法院关于互联网法院审理案件若干问题的规定》(以下简称《规定》)。《规定》第11条中明确规定:当事人提交的电子数据,通过电子签名、可信时间戳、哈希值校验、区块链等证据收集、固定和防篡改的技术手段或者通过电子取证存证平台认证,能够证明其真实性的,互联网法院应当确认。因此,区块链记录的电子证据可被认为是具有司法效力的证据,已有多个平台成功应用。

智能合同展新篇,社会合作换新颜:

区块链带来的智能合同实际上是在另一个物体的行动上发挥功能的计算机程序。与普通计算机程序一样,智能合同也是一种“如果—然后”的功能,但区块链技术实现了这些“合同”的自动填写和执行,无须人工介入。这种合同最终可能会取代法律行业的核心业务,即在商业和民事领域起草和管理合同的业务。

信息牢固于链连,政府透明社会安:

政务信息、项目招标等信息公开透明,政府工作通常受公众关注和监督,由于区块链技术能够保证信息的透明性和不可更改性,对政府透明化管理的落实有很大的作用。政府项目招标存在一定的信息不透明性,而企业在密封投标过程中也存在信息泄露的风险。区块链能够保证投标信息无法篡改,并能保证信息的透明性,在彼此不信任的竞争者之间形成信任共识。并能够通过区块链安排后续的智能合约,保证项目的建设进度,一定程度上防止了腐败的滋生。

\subsection{对经济体系的潜在变革}

供应链连商与户,经济监管减难度:

基于区块链的供应链金融应用中,通过将供应链上的每一笔交易和应收账款单据上链,同时引入第三方可信机构,例如银行、物流公司等,来确认这些信息,确保交易和单据的真实性,实现了物流、信息流、资金流的真实上链;同时,支持应收账款的转让、融资、清算等,让核心企业的信用可以传递到供应链的上下游企业,减小中小企业的融资难度,同时解决了机构的监管问题。

通过区块链进行数字资产交易,首先将链下资产登记上链,转换为区块链上的标准化数字资产,不仅能对交易进行存证,还能做到交易即结算,提高交易效率,降低机构间通信协作成本。监管机构加人联盟链中,可实时监控区块链上的数字资产交易,提升监管效率,在必要时进行可信的仲裁、追责。
