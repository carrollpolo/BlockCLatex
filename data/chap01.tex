% !TeX root = ../thuthesis-example.tex

\chapter{了解区块链}

本章开头用简短的篇幅介绍了区块链的来源,
以及理解接下来的章节关于区块链的一些必要知识。

\section{区块链的定义与原理}

区块链是一种基于对等网络(P2P网络)和现代密码学的安全可信的公共分布式数据库,具有公开透明、去中心化、匿名性以及不可篡改等特点。

区块链以块的形式存储数据,并使用密码学方法保证数据的安全性和完整性。每个块包含一定数量的交易信息,并通过加密链接到前一个块,形成一个不断增长的链条。这种设计使得数据在网络中无法被篡改,因为任何尝试修改一个块的数据都会破坏整个链的连续性。通过去中心化的网络结构,区块链技术实现了对数据的分布式共享和管理,从而在不需要信任中介的情况下确保了数据的安全和可靠性。



\section{区块链的特点}

\subsection{去中心化}
	
去中心化:在中本聪的设计中,每一枚比特币的产生都独立于权威中心机构,任意个人、组织都可以参与到每次挖矿、交易、验证中,成为庞大的比特币网络中的一部分。区块链网络通常由数量众多的节点组成,根据需求不同会由一部分节点或者全部节点承担账本数据维护工作,少量节点的离线或者功能丧失并不会影响整体系统的运行。在区块链中,各个节点和矿工遵守一套基于密码算法的记账交易规则,通过分布式存储和算力,共同维护全网的数据,避免了传统中心化机构对数据进行管理带来的高成本、易欺诈、缺乏透明、滥用权限等问题。普通用户之间的交易也不需要第三方机构介入,直接点对点进行交易互动即可 [1]。

\subsection{开放性}
开放性:区块链系统是开放的,它的数据对所有人公开,任何人都可以通过公开的接口查询区块链数据和开发相关应用,因此整个系统的信息高度透明。虽然区块链的匿名性使交易各方的私有信息被加密,但这不影响区块链的开放性,加密只是对开放信息的一种保护 。

在开放性的区块链系统中,为了保护一些隐私信息,一些区块链系统使用了隐私保护技术,使得人们虽然可以查看所有信息,但不能查看一些隐私信息[1] 。

\subsection{开放性}
匿名性:在区块链中,数据交换的双方可以是匿名的,系统中的各个节点无须知道彼此的身份和个人信息即可进行数据交换 。
	
我们谈论的隐私通常是指广义的隐私:别人不知道你是谁,也不知道你在做什么。事实上,隐私包含两个概念:狭义的隐私(Privacy)与匿名(Anonymity)。狭义的隐私就是别人知道你是谁,但不知道你在做什么;匿名则是别人知道你在做什么,但不知道你是谁。虽然区块链上的交易使用化名(Pseudonym),即地址(Address),但由于所有交易和状态都是明文,因此任何人都可以对所有化名进行分析并建构出用户特征(User Profile)。更有研究指出,有些方法可以解析出化名与IP的映射关系,一旦IP与化名产生关联,则用户的每个行为都如同裸露在阳光下一般。

在比特币和以太坊等密码学货币的系统中,交易并不基于现实身份,而是基于密码学产生的钱包地址。但它们并不是匿名系统,很多文章和书籍里面提到的数字货币的匿名性,准确来说其实是化名。在一般的系统中,我们并不明确区分化名与匿名。但专门讨论隐私问题时,会区分化名与匿名。因为化名产生的信息在区块链系统中是可以查询的,尤其是在公有链中,可以公开查询所有的交易的特性会让化名在大数据的分析下完全不具备匿名性。但真正的匿名性,如达世币、门罗币、Zcash等隐私货币使用的隐私技术才真正具有匿名性。

匿名和化名是不同的。在计算机科学中,匿名是指具备无关联性(Unlinkability)的化名。所谓无关联性,就是指网络中其他人无法将用户与系统之间的任意两次交互(发送交易、查询等)进行关联。在比特币或以太坊中,由于用户反复使用公钥哈希值作为交易标识,交易之间显然能建立关联。因此比特币或以太坊并不具备匿名性。这些不具备匿名性的数据会造成商业信息的泄露,影响区块链技术的普及使用[1] 。
	
\subsection{可追溯性}
可追溯性:区块链采用带时间戳的块链式存储结构,有利于追溯交易从源头状态到最近状态的整个过程。时间戳作为区块数据存在的证明,有助于将区块链应用于公证、知识产权注册等时间敏感领域 [1]。

\subsection{透明性}
透明性:相较于用户匿名性,比特币和区块链系统的交易和历史都是透明的。由于在区块链中,账本是分发到整个网络所有参与者,账本的校对、历史信息等对于账本的持有者而言,都是透明的、公开的[1] 。

\subsection{不可篡改性}
不可篡改性:比特币的每次交易都会记录在区块链上,不同于由中心机构主宰的交易模式,其中心机构可以自行修改任意用户的交易信息,比特币很难篡改[1] 。

\subsection{多方共识}
多方共识:区块链作为一个多方参与维护的分布式账本系统,参与方需要约定数据校验、写入和冲突解决的规则,这被称为共识算法。比特币和以太坊作为公有链当前采用的是工作量证明算法(PoW),应用于联盟链领域的共识算法则更加灵活多样,贴近业务需求本身[1]。 








